\documentclass{article}
\usepackage{amsmath, amsthm, amssymb}
\usepackage{graphicx}
\usepackage{verbatim}
\usepackage{natbib}
\usepackage{caption}
\usepackage{subcaption}
\usepackage{fancyvrb}
\usepackage{enumerate}
\usepackage{relsize}
\usepackage{multirow}
\usepackage[section]{placeins}

\usepackage{hyperref}
\usepackage[margin=1.4in]{geometry}
\hypersetup{colorlinks,citecolor=blue,urlcolor=blue,linkcolor=blue}

% \newcommand\blfootnote[1]{%
%   \begingroup
%   \renewcommand\thefootnote{}\footnote{#1}%
%   \addtocounter{footnote}{-1}%
%   \endgroup
% }

\usepackage{stefan_tex}
\graphicspath{{figs/}}

%\numberwithin{equation}{section}

%%%%%%%%% Theorems
\theoremstyle{plain}
\newtheorem{prop}{Proposition}

\newtheorem{conj}[prop]{Conjecture}
\newtheorem{coro}[prop]{Corollary}
\newtheorem{lemm}[prop]{Lemma}
\newtheorem{theo}[prop]{Theorem}

\theoremstyle{definition}
\newtheorem{exam}{Example}
\newtheorem{defi}[exam]{Definition}
\newtheorem{assumption}{Assumption}
\newtheorem{property}{Property}

\theoremstyle{remark}
\newtheorem{comm}[prop]{Comment}
\newtheorem{rema}[prop]{Remark}

\newcommand\blfootnote[1]{%
  \begingroup
  \renewcommand\thefootnote{}\footnote{#1}%
  \addtocounter{footnote}{-1}%
  \endgroup
}

\newcommand{\amy}[1]{\color{red}{\emph{#1} - amy}\color{black}}

\author{
  Amy Guan$^{1}$ \and
  Roshni Sahoo$^{1}$ \and 
  Josh Salomon$^{2}$ \and
  Stefan Wager$^{3}$ \and
  Marissa Reitsma$^{2}$
}

\date{Stanford University}


\title{High-Resolution Estimates of
Population Health Indicators from Large-Scale Online
Surveys with Sampling Bias}

\begin{document}


\maketitle

\begin{abstract}
High-resolution estimates of population health indicators are necessary for targeted and equitable public health interventions. 
 L\blfootnote{\hspace{-5.3mm}{Draft version: \ifcase\month\or January\or February\or March\or April\or May\or June\or July\or August\or September\or October\or November\or December\fi \ \number \year }. $^{1}$ Department of Computer Science, $^{2}$ Department of Health Policy, $^{3}$ Graduate School of Business.}arge-scale online surveys are a promising tool for obtaining these estimates because they have low implementation costs and can scale to large sample sizes, but they often suffer from sampling bias, meaning the population of survey respondents may differ from the target population.
\end{abstract}

\begin{section}{Introduction}

% What is the problem? 

% We want high resolution estimates of population health outcomes across spatial strata.

% Why is it interesting and important?

% Why is it hard? (E.g., why do naive approaches fail?)
% Why hasn't it been solved before? (Or, what's wrong with previous proposed solutions? How does mine differ?)
% What are the key components of my approach and results? Also include any specific limitations.

High resolution estimates of population health outcomes across spatial strata are necessary to inform targeted and equitable public health interventions. \rs{elaborate more on why this is useful}

Traditional representative in-person household surveys are the gold standard for public health data collection. While these surveys are representative at a \textit{macro-geographic} level (i.e. national level), they cannot currently be used to produce \textit{micro-geographic} level (i.e. state or county level) estimates due to limited sample size and high deployment costs.

Large-scale online survey are a promising tool for obtaining micro-geographic level health estimates \citep{geldsetzer2020use, us2021measuring, salomon2021us} because compared to traditional in-person surveys, online surveys are faster to deploy, cheaper to administer, and scale to much larger sample sizes \citep{blumberg2021national}. However, their key limitation is that they suffer heavily from sampling bias, meaning that the population of survey respondents differs from the target population. Without bias adjustment, conclusions drawn from online surveys may not be representative and generalizable.

Sampling bias in online surveys is typically addressed by applying covariate-based reweighting methods, such as inverse propensity weighting and post-stratification \citep{ groves2011survey, rosenbaum1983central}. These techniques yield unbiased estimation of the target outcome if the sampling bias is \textit{ignorable} \citep{rubin1976inference}, meaning that an individual's probability of survey response depends only on observable covariates. However, recent empirical analyses find that these techniques fail to sufficiently adjust for the bias in online surveys \citep{bradley2021unrepresentative, kessler2022estimated}, indicating that sampling bias in online surveys likely \textit{non-ignorable}. Under non-ignorable sampling bias, an individual's probability of survey response can depend on unobservable covariates that also affect their outcome, as well as observed covariates. For instance, \citet{kessler2022estimated} find that an online survey called the Household Pulse Survey (HPS) yields implausibly high estimates of the prevalence of anxiety-depression compared to a telephone survey called the Behavioral Risk Factor Surveillance System (BRFSS). They observe that the HPS respondent population differs from the target population in geographic-demographic characteristics, and they hypothesize that the two populations differ in their unmeasured psychological characteristics, as well.

The main goal of this work is to evaluate whether online surveys, with appropriate bias adjustment, are a cost-effective tool for obtaining timely, micro-geographic level population health estimates. To this end, our first contribution is a simple and practical method for adjusting for non-ignorable sampling bias in online surveys. We provide a procedure for learning micro-geographic level estimates of population health outcomes from potentially biased, individual level online survey data and macro-geographic level target outcome data, which is assumed to be unbiased. The key assumption underlying our approach is that sampling bias in online surveys satisfies the conditional $\Gamma$-biased sampling model \citep{sahoo2022learning}, which posits that observed covariates can affect the probability of sample selection arbitrarily much but the amount of unexplained variation in the probability of sample selection is bounded by a constant. Under this assumption, we learn partial identification bounds on the conditional mean outcome in the target population from the online survey data for a range of bias-adjustment parameters $\Gamma$, and we calibrate $\Gamma$ using the macro-geographic level target data. Our approach is compatible with existing post-stratification and importance reweighting techniques that correct for sampling bias due to observed covariates.

Our second contribution is the development of a rich testbed for evaluating the quality of micro-geographic level estimates obtained from online survey data. Evaluating the quality of estimates obtained from online surveys is often infeasible because ground truth measurements of the outcomes in the target population are unavailable. Nevertheless, we identify three different population health indicators for which there are repeated measurements from an online survey and a ground truth data source for the same target population over the same period of time (see Table \ref{tab:data_pairing}). Furthermore, for each indicator, we identify ground truth data sources that contain measurements of the indicators at two geographic resolutions, a macro-geographic level and a micro-geographic level. The online surveys included in our testbed are HPS \citep{us2021measuring} and the US COVID-19 Trends \& Impact Survey (CTIS) \citep{salomon2021us}. We evaluate the performance of our method as well as other baselines on our testbed and find that across different indicators and online surveys, our method yields lower error in micro-geographic level estimates than methods that only utilize the online survey data or target macro-geographic level data alone. \rs{quantify the amount of benefit that we get}

Our last contribution is an analysis of the cost-accuracy tradeoffs of online surveys versus traditional houshold surveys. For the COVID-19 vaccination indicator, we find that online surveys with bias adjustment can obtain the same level of error as traditional offline surveys at \rs{some fraction} of the cost \rs{provide some quantitative takeaway about the tradeoffs}.

% Our approach assumes that sampling bias in the online survey satisfies the conditional $\Gamma$-biased sampling model \citep{sahoo2022learning}. Under conditional $\Gamma$-biased sampling, observed covariates can affect an individual's probability of sample selection arbitrarily much but the amount of unexplained variation in the
% probability of sample selection is bounded by a constant factor.


	\begin{table}[ht]
    \centering
    \renewcommand{\arraystretch}{1.5} % Adjust row height for better readability
    \setlength{\tabcolsep}{6pt} % Adjust column spacing
    \begin{tabular}{|c|c|c|c|c|}
        \hline
        \multirow{3}{5em}{Health Indicator} & \multirow{3}{5em}{Online Survey} & \multirow{3}{5em}{Ground Truth (GT)} & \multirow{3}{5em}{GT Macro-Geographic Level} & \multirow{3}{5em}{GT Micro-Geographic Level}\\
        & & & &  \\
        & & & &  \\
        \hline 
        COVID-19 Vaccination & HPS  & CDC    & National & State  \\
        COVID-19 Vaccination & CTIS & CDC    & State    & County \\
        SNAP Enrollment      & HPS  & USDA   & National & State  \\
        Smoking Prevalence   & CTIS & BRFSS  & State    & Metropolitan Division \\
        Mental Health        & HPS  & BRFSS  & National & State \\ 
        \hline
    \end{tabular}
    \caption{Our testbed spans four different population health indicators, two online surveys, and target data sources at two levels of geographic resolution. Our learning procedure utilizes the online survey data and target data at the macro-geographic level to provide micro-geographic level estimates. The target micro-geographic level data is used to evaluate these estimates.}
    \label{tab:data_pairing}
\end{table}


\begin{subsection}{Related Work} 
	Most prior works that aim to correct for sampling bias in survey data assume that sampling bias is ignorable, which means that an individual's probability of survey response only depends on observable characteristics but do not correct for non-ignorable selection. This type of sampling bias can be addressed via post-stratification \citep{groves2008impact, groves2011survey, little1993post}  or propensity score reweighting \citep{david1983imputation,rosenbaum1983central,seaman2013review}.

	However, empirical studies of recent online surveys suggest that sampling bias in online surveys is non-ignorable \citep{bradley2021unrepresentative, kessler2022estimated}. Many previous works aim to address non-ignorable sampling bias in surveys \citep{andridge2011proxy, andridge2019indices, greenlees1982imputation, little2020measures, manski2016credible, qin2002estimation, peress2010correcting, reitsma2022bias, wisniowski2020integrating}. 

\rs{roshni to spend time reviewing the literature}
\begin{enumerate}
    \item \citet{peress2010correcting} aims to correct for non-response by estimating units' response propensity from proxies such as number of attempted phone calls, indicators of temporary refusal, and interviewer-coded measures of cooperativeness and extrapolating from the low-propensity respondents to learn about the nonrespondent population. 

    \item \citet{greenlees1982imputation,qin2002estimation} account for non-ignorable sampling bias using parametric and semiparametric modeling approaches \rs{todo add detail}. 

    \item \citet{reitsma2022bias} learns a mapping from micro-geographic level estimates from an online survey to micro-geographic level target data for a single indicator and uses this mapping to correct for bias in other indicators. 

    \item \citet{wisniowski2020integrating} proposes a Bayesian modeling approach \rs{todo add detail}. Our work differs from these previous because we do not assume access to proxy variables, access to micro-geographic level target data, or a parametric form for the probability of survey response.

    \item \citet{andridge2011proxy,andridge2019indices,little2020measures} - Gaussian latent modeling approach

    \item \citet{manski2016credible} Manski bounds where nonselected individuals either have $Y=1$ or $Y=0$
\end{enumerate}

	Instead, the key assumption underlying our approach is the conditional $\Gamma$-biased sampling model, which is nonparametric model for non-ignorable sampling bias \citep{sahoo2022learning}. \citet{sahoo2022learning} also consider learning under this model, but their goal is to learn a predictive model that minimizes the worst-case risk under $\Gamma$-biased sampling, not to obtain partial identification bounds on micro-geographic level target quantities. Furthermore, they also leave the question of how to select $\Gamma$ largely open, while our procedure provides concrete guidance on this choice by leveraging macro-geographic level target outcome data to select $\Gamma$. The form of the partial identification bounds under conditional $\Gamma$-biased sampling can be obtained by applying the results of \citet{dorn2024doubly,rockafellar2000optimization}.
\end{subsection}






\end{section}

\begin{section}{Methods}
\label{sec:method}
We give the problem setup and describe how to compute partial identification bounds on the micro-geographic level outcomes in the target population.

\begin{subsection}{Problem Setup}
Let $P$ denote the data distribution over units' observed covariates and health outcomes in the population of online survey respondents. Let $\mathcal{X}, \mathcal{Y}$ be the space of covariates and outcomes, respectively. We observe individual level survey data in the form of $i=1,2, \dots n$ samples $(X_{i}, Y_{i})$ independently drawn from $P$, where $X_{i} \in \mathcal{X}$ are covariates and $Y_{i} \in \mathcal{Y}$ is the health outcome. We assume that the observed covariates $X_{i}$ contains the micro- and macro-geographic region that unit $i$ belongs to. 

Let $Q$ denote the data distribution over units' observed covariates and health outcomes in the target population. We assume that the target covariate distribution $Q_{X}$ is known; in our applications, $Q_{X}$ is given by the ACS Community Survey (US Census) \citep{bureau2021american}. Note that because the online survey suffers from sampling bias, $Q$ likely differs from $P$. Let $\mathcal{G}$ be the set of macro-geographic regions. For $g \in \mathcal{G}$, let $\mathcal{X}_{g}$ denote the subset of the covariate space that corresponds to individuals in macro-geographic region $g$. We observe macro-geographic level target outcomes, i.e. we see $\mu_{g} := \EE[Q]{Y \mid X \in \mathcal{X}_{g}} $ for all $g \in \mathcal{G}.$ 

Let $\mathcal{H}$ be the set of micro-geographic regions. For $h \in \mathcal{H},$ let $\mathcal{X}_{h}$ denote the subset of the covariate space that corresponds to individuals in micro-geographic region $g$. The target quantities that we are interested in are $\mu_{h}:= \EE[Q]{Y \mid X \in \mathcal{X}_{h}}$ for all $h \in \mathcal{H}.$

Note that with access to $\mu_{Q}(x):= \EE[Q]{Y \mid X=x}$, the conditional mean outcome in the target population, it is straightforward to estimate $\mu_{h}$ for $h \in \mathcal{H}.$ However, we only observe individual level data from $P$ and macro-geographic level outcome data from $Q$, so $\mu_{Q}(\cdot)$ is not identifiable.

Central to our approach, we assume that the survey population $P$ is generated from the target population $Q$ via conditional $\Gamma$-biased sampling for $\Gamma \geq 1$ \citep{sahoo2022learning}. In $\Gamma$-biased sampling, observable covariates $X_{i}$ can affect a unit's probability of survey response arbitrarily, while the amount of unexplained variation in the probability of sample selection is bounded by a coonstant factor. \citet{sahoo2022learning} demonstrate that if $Q$ generates $P$ via conditional $\Gamma$-biased sampling, then
\begin{equation} \Gamma^{-1} \leq \frac{dQ_{Y \mid X=x}(y)}{dP_{Y \mid X=x}(y)} \leq \Gamma \quad \forall x \in \mathcal{X}, y \in \mathcal{Y}.\end{equation}
We note that when $\Gamma = 1$, this model reduces to the standard assumption of ignorable sampling bias. As $\Gamma$ increases, the amount of non-ignorable sampling bias permitted and the amount that the online survey conditional distribution can deviate from the target conditional distribution grows. Let $\mathcal{S}_\Gamma(P, Q_X)$ to be the set of all distributions $Q$ that satisfy the assumptions of $\Gamma$-biased sampling for an online survey distribution $P$ and a particular covariate distribution $Q_X$.

\end{subsection}

\begin{subsection}{Partial Identification Bounds}
We note that $\mu_{h}$ for $h \in \mathcal{H}$ is not identifiable. Nevertheless, we can estimate partial identification bounds on these quantities. First, we propose to estimate partial identification bounds on the target conditional mean outcome $\mu_{Q}(\cdot)$ under the conditional $\Gamma$-biased sampling model. Then, we use our partial identification bounds on $\mu_{Q}(\cdot)$ to obtain bounds on $\mu_{h}$ for $h \in \mathcal{H}.$

 If $Q$ generates $P$ under conditional $\Gamma$-biased sampling and has covariate distribution $Q_{X}$, then $Q$ lies in the set of distributions $\mathcal{S}_{\Gamma}(P, Q_{X}).$ So, the partial identification bounds on $\mu_{Q}(x)$ are given by 
\begin{equation}
\label{eq:def_bound}
\mu_{\Gamma}^{-}(x):= \inf_{Q \in \mathcal{S}_{\Gamma}(P, Q_{X})} \EE[Q]{Y \mid X=x}, \quad  \mu_{\Gamma}^{+}(x):= \sup_{Q \in \mathcal{S}_{\Gamma}(P, Q_{X})}. \EE[Q]{Y \mid X=x}.
\end{equation}
So, if $Q$ generates $P$ via conditional $\Gamma$-biased sampling, then we must have that $\mu_{\Gamma}^{-}(x) \leq \mu_{Q}(x) \leq \mu_{\Gamma}^{+}(x)$ for all $x \in \mathcal{X}.$

Let $q_{\alpha}(x)$ denote the conditional $\alpha$-quantile function, i.e. $q_{\alpha}(x) = P_{Y \mid X=x}^{-1}(\alpha).$ Then we can apply an argument analogous to Proposition 3 of \citet{dorn2024doubly} to obtain closed-form expressions for $\mu_{\Gamma}^{-}(\cdot), \mu_{\Gamma}^{+}(\cdot)$ as follows
\begin{equation}
\label{eq:bounds}
\begin{split}
\mu_{\Gamma}^{+}(x) &= \Gamma^{-1}\EE[P]{Y \mid X=x} + (\Gamma - \Gamma^{-1}) \EE[P]{Y \cdot \mathbb{I}(Y \geq q_{\eta(\Gamma)}(X)) \mid X=x}\\
\mu_{\Gamma}^{-}(x) &= \Gamma \EE[P]{Y \mid X=x} + (\Gamma^{-1} - \Gamma) \cdot \EE[P]{Y \cdot \mathbb{I}(Y \geq q_{\zeta(\Gamma)}) \mid X=x},
\end{split}
\end{equation}
 $\eta(\Gamma) = \frac{\Gamma}{\Gamma + 1}$ and $\zeta(\Gamma) = \frac{1}{\Gamma + 1}.$ Note that when $\Gamma = 1$, then both bounds reduce to the conditional mean of the online survey population $P$. We note that the bounds in \eqref{eq:bounds} depend on a conditional quantile function and moments of $P$, which are straightforward to estimate from the online survey data. A detailed estimation procedure is provided in Appendix \ref{app:alg}.

When $Y$ is binary valued, i.e. $\mathcal{Y} = \{0, 1\}$, these partial identification bounds can be further simplified to
\begin{equation} 
\label{eq:binary_bounds}
\begin{split}
\mu_{\Gamma}^{+}(x) &= \min\{1 - \Gamma^{-1} + \EE[P]{Y \mid X=x} \cdot \Gamma^{-1},\, \EE[P]{Y \mid X=x}  \cdot \Gamma\} \\
\mu_{\Gamma}^{-}(x) &= \max\{ 1 -  \Gamma + \EE[P]{Y \mid X=x} \cdot \Gamma,\, \EE[P]{Y \mid X=x}  \cdot \Gamma^{-1} \}.
\end{split}
\end{equation}
We note that these partial identification bounds are a simple transformation of the conditional mean outcome in the online survey population.

We emphasize that the bounds in \eqref{eq:bounds} and \eqref{eq:binary_bounds}, are not new to the literature; \citet{dorn2024doubly} derives a similar result in the context of obtaining partial identification bounds on causal effects in the presence of unmeasured confounding. Nevertheless, to the best of our knowledge, these partial identification bounds have not been applied to the problem of obtaining high-resolution estimates from online surveys with sampling bias.

Given partial identification bounds on $\mu_{Q}(\cdot)$, we can compute partial identification bounds on $\mu_{h}$ for $h \in \mathcal{H}.$ Note that
\begin{equation}
\mu_{h} = \frac{\EE[Q_{X}]{\mu_{Q}(X) \cdot \mathbb{I}(X \in \mathcal{X}_{h}) }}{\PP[Q_{X}]{X \in \mathcal{X}_{h}}},
\end{equation}
which depends on the known target covariate distribution $Q_{X}$ and the unknown target conditional mean outcome $\mu_{Q}(\cdot)$. Given our partial identification bounds on $\mu_{Q}(\cdot)$, for any $h \in \mathcal{H}$, we can compute partial identification bounds on $\mu_{h}$ such that $\mu_{h, \Gamma}^{-} \leq \mu_{h} \leq \mu_{h, \Gamma}^{+}$, where
\begin{equation}
\begin{split}
\mu_{h, \Gamma}^{-} = \frac{\EE[Q_{X}]{\mu_{\Gamma}^{-}(X) \cdot \mathbb{I}(X \in \mathcal{X}_{h}) }}{\PP[Q_{X}]{X \in \mathcal{X}_{h}}},\, \quad \mu_{h, \Gamma}^{+} &= \frac{\EE[Q_{X}]{\mu_{\Gamma}^{+}(X) \cdot \mathbb{I}(X \in \mathcal{X}_{h}) }}{\PP[Q_{X}]{X \in \mathcal{X}_{h}}}.
\end{split}
\end{equation}
Note that we can define analogous quantities $\mu_{g, \Gamma}^{-}, \mu_{g, \Gamma}^{+}$ to obtain partial identification bounds on $g \in \mathcal{G}.$Thus, if we posit that the online survey population $P$ is related to the target population $Q$ through conditional $\Gamma$-biased sampling, we can obtain partial identification bounds on $\mu_{h}$ for $h \in \mathcal{H}$ that depend on $\Gamma$.
\end{subsection}

\begin{subsection}{Heuristic for Point Estimation}
\label{subsec:heuristic}
We can use the procedure from the previous subsection to compute partial identification bounds for $\mu_{g}$ such that $g \in \mathcal{G}$ for various choices of $\Gamma \geq 1.$ Note that if $Q$ generates $P$ via conditional $\Gamma$-biased sampling for a particular value of $\Gamma$, then the true target outcome $\mu_{g}$ will satisfy the consistency condition $\mu_{g, \Gamma}^{-} \leq \mu_{g} \leq \mu_{g, \Gamma}^{+}.$ Thus, without any further assumptions, one can calibrate $\Gamma$ by finding the minimum value of $\Gamma$ that ensures the consistency condition holds for all $g \in \mathcal{G}.$ Then this value of $\Gamma$ can be used to obtain partial identification bounds for $\mu_{h}$ for $h \in \mathcal{H}.$ However, one limitation of this approach is that the partial identification bounds obtained may be quite wide.

Alternatively, if we are willing to further assume that the target population $Q$ is a worst-case distribution in $\mathcal{S}_{\Gamma}(P, Q_{X})$ for some $\Gamma \geq 1$, i.e. $Q$ is the distribution that yields the infimum or the supremum in \eqref{eq:def_bound} for some $\Gamma$, then we can propose a procedure for calibrating $\Gamma$ and obtaining point estimates for $\mu_{h}$. Note that if $Q$ yields the infimum in \eqref{eq:def_bound}, then we have that $\mu_{Q}= \mu_{\Gamma}^{-}$ and $\mu_{g} = \mu_{g, \Gamma}^{-}$ for all $g \in \mathcal{G}$ for some $\Gamma$. An analogous argument applies if $Q$ is the maximizing worst-case distribution in $\mathcal{S}_{\Gamma}(P, Q_{X})$--in this case, $\mu_{g} = \mu_{g, \Gamma}^{+}$ for all $g \in \mathcal{G}$ for some $\Gamma$. If the target population $Q$ is a worst-case distribution in $\mathcal{S}_{\Gamma}(P, Q_{X})$, then we can use either the upper or lower partial identification bound as a point estimate and select $\Gamma$ and the bound to use by minimizing the MSE between the target macro-geographic level outcomes and the partial identification bounds on the macro-geographic level outcomes:
\begin{equation} (\tilde{\Gamma}, \tilde{z}) \in \text{argmin}_{\Gamma \geq 1, z \in \{+, -\}} \frac{1}{|\mathcal{G}|} \sum_{g \in \mathcal{G}} (\mu_{g, \Gamma}^{z} - \mu_{g})^{2}.\end{equation}

We can use $(\tilde{\Gamma}, \tilde{z})$ obtained from solving this minimization problem to a compute micro-geographic level point estimate of $\mu_{h}$ as follows
\begin{equation} \tilde{\mu}_{h} = \frac{\EE[Q_{X}]{\mu_{\tilde{\Gamma}}^{\tilde{z}}(X) \cdot \mathbb{I}(X \in \mathcal{X}_{h})}}{\PP[Q_{X}]{X \in \mathcal{X}_{h}}}.\end{equation}

\rs{is there any way to strengthen this section?? more justification for using the bounds as point estimates?}

\end{subsection}



\end{section}

\begin{section}{Experiments}
We evaluate quality of the micro-geographic level estimates obtained from online surveys using our testbed (Table \ref{tab:data_pairing}).

\begin{subsection}{Temporal Evaluation Setup}
Both the online surveys and ground truth data sources of our test bed are collected at frequent time intervals (see Appendix \ref{app:dataset} for details), so we evaluate the micro-geographic estimates obtained from online surveys at multiple time steps $t=1, 2, \dots, T$. Let $P^{t}, Q^{t}$ denote the online survey population and target population at time step $t$, respectively. For simplicity, in our experiments, the target covariate distribution $Q_{X}^{t}$ is given by the 2020 Census and does not change with $t$. However, note that the online survey distribution $P^{t}$ and the target conditional distribution $Q_{Y|X}^{t}$ may evolve of time.

To generate micro-geographic level estimates at time step $t$, all methods only utilize data from time step $t$. We evaluate two baselines and our approach:
\begin{enumerate}
\item Target Macro-Geographic: Target macro-geographic outcomes from the ground truth data source ared used as micro-geographic level estimates. In other words, if micro-geographic region $h$ lies in macro-geographic region $g$, this approach sets the estimate of the mean target outcome in $h$ to $\mu_{g}$.
\item Naive Micro-Geographic: Online survey data and post-stratification weights at time step $t$ are used generate micro-geographic level estimates. Note that this approach is equivalent to running our method with $\Gamma=1$ and does not adjust for non-ignorable sampling bias.
\item Bias-Adjusted Micro-Geographic: We use the procedure described in Section \ref{sec:method} to generate micro-geographic level estimates from the online survey data from time step $t$ and macro-geographic level ground truth data from time step $t$. 
\end{enumerate}
Importantly, none of these approaches rely on the target micro-geographic level outcomes at time step $t$, which are used purely for evaluation.

For all methods, we report the MAE (mean absolute error) of the micro-geographic level estimates at time step $t$, i.e.
\[  \text{MAE}(t):= \frac{1}{|\mathcal{H}|} \sum_{h \in \mathcal{H}} | \hat{\mu}_{h}^{t} - \mu_{h}^{t}|.\]
\end{subsection}

\begin{subsection}{Results}

\begin{subsubsection}{COVID-19 Vaccination}
\rs{can be made more concise!}
We find that the bias adjustment parameter selected by our procedure in Section \ref{subsec:heuristic} changes gradually over time (Figure \ref{fig:arg_gamma}). For both online surveys, the optimal bias adjustment parameter $\Gamma$ exhibits a similar trend, where the optimal bias adjustment increases from April to June 2021, when vaccination coverage was increasing rapidly, then slightly decreases in July 2021, when vaccination coverage began to plateau. 

We also evaluate the performance of our bias-adjustment approach. For both online surveys, we see a marked improvement in performance with bias adjustment, where we are able to maintain a low, relatively constant MAE when we select $\Gamma$ with the procedure from Section \ref{subsec:heuristic}. We also observe that this method outperforms the naive approach that relies on the online survey data without adjusting for non-ignorable sampling bias and also outperforms using the macro-geographic level target outcomes as micro-geographic level estimates.

%  %This verifies that the sampling bias model is effective at capturing a significant portion of the sampling bias within the HPS survey.

% For CTIS, we compute the state-level MSE between the county-level partial-identification estimates and CDC ground truths with and without bias adjustment (Figure \ref{fig:ctis}, top). With an optimized bias-adjustment ($\hat{\Gamma}$), selected using state-level ground truth data, we are again able to improve small-area estimates. We also compute the MSE from using the state-level ground truth as the county-level estimate (Figure \ref{fig:ctis}, bottom). We see that this substitution outperforms the case where we regress without adjusting for sampling bias ($\Gamma=1$) but does not outperform an optimized bias-adjustment. This suggests that we may improve surveillance systems via this bias sampling model by simply incorporating both online surveys and traditional data sources.
\end{subsubsection}

\begin{subsubsection}{SNAP and Smoking}
\rs{amy to add}

\end{subsubsection}

\end{subsection}


\end{section}

\begin{section}{Discussion}
This provides insight into how difficult it would be to perform accurate bias adjustment with lagged ground-truth data for a target outcome like COVID-19 vaccination outcomes that are expected to experience a high amount of change over a time. This indicates that future surveillance systems for such outcomes would still benefit from a mixture of online surveys and traditional data sources for calibration.

\end{section}


\bibliographystyle{plainnat} 
\bibliography{ref.bib}

\appendix

\begin{section}{Estimation Procedure}
\label{app:alg}
\input{05-estimation}
\end{section}

\begin{section}{Testbed}
\label{app:dataset}
We outline different components of our testbed.
\begin{subsection}{Online Survey Data Sources}
We utilize data from two of the largest recent online surveys containing public health data (Table \ref{tab:surveys}). From each survey, we have access to individual level health indicators; the indicators available in each survey are outlined in Table \ref{tab:data_pairing}. The online surveys also have individual level demographic features, including age, gender, race/ethnicity, income, education level, and state or county identifiers. 
\begin{enumerate}
\item \textbf{Household Pulse Survey (HPS)} \citep{us2021measuring} - HPS is a survey that has been administered biweekly by the US Census Bureau since the beginning of the COVID-19 pandemic. The survey focuses on health, economic, and social topics, with questionnaires evolving over time. The HPS data are publicly available with state-level identifiers. We restrict our analysis to HPS data collected 11 weeks after vaccine emergency use authorization (EUA) for adults from February 23, 2021 - April 20, 2021 \rs{is this true for SNAP as well?}.

\item \textbf{US COVID-19 Trends and Impact Survey (CTIS)} \citep{salomon2021us} - CTIS is a survey administered by Facebook and Carnegie Mellon University that ran daily from April 6, 2020, to June 25, 2022 and aimed to record the spread and impact of COVID-19 in the United States. Deidentified individual level data are available to researchers. CTIS was administered more frequently and had a larger sample size than HPS. Additionally, CTIS data included county-level identifiers. We restrict our analysis to CTIS data collected between March and August 2021.
\end{enumerate}

\begin{table}[ht]
    \centering
    \renewcommand{\arraystretch}{1.5} % Adjust row height for better readability
    \setlength{\tabcolsep}{6pt} % Adjust column spacing
    \begin{tabular}{|c|c|c|c|c|c|}
        \hline
        \multirow{3}{4em}{Online Survey} & \multirow{3}{4em}{Total Sample Size} & \multirow{3}{4em}{Frequency} & \multirow{3}{6em}{Sampling Frame} & \multirow{3}{5em}{Geographic Resolution} & \multirow{3}{5em}{Administrator}\\
        & & & & & \\
        & & & & & \\
        \hline 
        HPS & 5 million & \multirow{3}{4em}{Every two weeks}    & \multirow{3}{6em}{US Census Bureau's Master Address File} & State & US Census Bureau  \\
        & & & & & \\
        & & & & & \\
        \hline
        CTIS & 29 million & Daily    & \multirow{2}{6em}{Facebook Active Users}   & County & Facebook, CMU\\
        & & & & & \\
        \hline
    \end{tabular}
    \caption{Our testbed includes two large-scale online surveys, HPS and CTIS. }
    \label{tab:surveys}
\end{table}
\end{subsection}

\begin{subsection}{Ground Truth Covariate Distribution} 
\noindent\textbf{American Communities Survey (ACS)} - \rs{add description of ACS survey}
\end{subsection}


\begin{subsection}{Ground Truth Data Sources}

\noindent\textbf{CDC Ground-Truth Vaccination Rates} \rs{add citation?}- Ground-truth vaccination rates at the state-level and county-level across time were collected by the CDC and publicly available \citep{covid_data}. We use these data to evaluate bias in estimates of coverage from HPS and CTIS, to calibrate $\Gamma$, and to evaluate predictions.


\rs{need to add USDA} 
\newline
\rs{need to add BRFSS}

\rs{Add a discussion (maybe to the main text?) on the extent to which these datasets are ``ground truth''/gold standard, e.g. discuss that BRFSS is itself a telephone (but Kessler et al treat it like it is ground truth).}
\end{subsection}

\end{section}

\begin{section}{Training Details}
\label{app:training}

\rs{amy to resolve comments in this section.}
\subsection{Features and Dataset Preprocessing}
\rs{Outline for this section:
\begin{enumerate}
	\item For HPS, we obtain individual level features including...
	\item Then say: We supplement the individual level features available from the online surveys with area level features obtained from the ACS Community Survey that correspond to each individual's state identifier.
	\item Then describe state level features that we merge in
	\item Repeat for CTIS. For CTIS, we obtain individual level features including...
\end{enumerate}}


For each HPS respondent, we consider the covariates sex, race, age group, education category, insurance status, and household size. For sex, race, and insurance status, we one-hot encode the demographic information. For the categorical variables age group, education level, and household size, we assign ordered integer values to each category. We exclude any of the surveyed who did not respond to whether or not they have received a COVID-19 vaccination.

To enable more information sharing among similar states, we also augment each respondent's covariates with the ACS state characteristics of their respective state \amy{is this from 2021?}. These characteristics include demographic information, such as median income, political party rates, average household size, and more \amy{do i need to include a full list?}.\rs{yes add full list!}

\subsection{Post-Stratification Weights}
\rs{I copied and pasted the relevant sections that we previously wrote about this. Needs clean up.}
We estimate the target covariate distribution $Q_{X}$ using data from the 2020 Census. From the Census, we can estimate $dQ_{X}$ for every combination of FIPS code, sex, age category, education, and race group in the United States. We implement this by following the procedure in \citet{royal2019survey}.

For each time window $t$, we compute our training covariate distribution $P_{X}^{t}$ using the 2021 CTIS data collected in time window $t$. We can estimate $dP_{X}^{t}$ for every combination of FIPS code, sex, age category, education, and race group in the dataset by computing their empirical proportions.

With these estimates of $dQ_{X}$ and $dP^{t}_{X}$, we can compute the post-stratification weights as follows
\[ r^{t}(x) = \frac{dQ_{X}^{t}(x)}{dP_{X}^{t}(x)} = \frac{dQ_{X}(x)}{dP^{t}_{X}(x)}.\]

Using the ACS 2020 population survey, we generate post-stratification weights per demographic feature combination across age, gender, race/ethnicity, income, education level, and geographic region (state for HPS, county for CTIS).

For each state, we sum the ACS weights across each set of covariates to compute a post-stratification weight for each respondent. These are used within both the model fitting step and the final state-wide aggregation when obtaining the final prediction.

\subsection{Learning Procedure}
We note that all of the population health outcomes in our testbed are binary. So, the partial identification bounds can be obtained by \eqref{eq:binary_bounds}. 

\textbf{Models.}
We estimate $\mu_{P^{t}}(x):= \EE[P^{t}]{Y \mid X=x}$ for each time-step $t$ via logistic regression using the default implementation in the scikit-learn package.

\textbf{Dataset splits.} At each time-step $t$, we split the data from the online survey into a 60\% train, 20\% validation, 20\% test split. The average number of samples in the HPS dataset per week is $n= \rs{todo}$, which yields \rs{todo} training samples, \rs{todo} validation samples, and \rs{test} samples per time step. The average number of samples in the CTIS dataset per week is $n=\rs{todo}$, which yields \rs{todo} training samples, \rs{todo} validation samples, and \rs{test} samples.

\rs{can you specify what each dataset is used to do. For example, we fit the logistic regression model on the training set. We use the validation set for early stopping. We use the test set to compute the partial identification bounds.}

\textbf{Training Procedure.}

For each timestep, we estimate $\mu_{P^{t}}(\cdot)$ on the train and validation sets of the processed HPS survey data. 

\rs{explain how state/county level estimates are computed and using what data}

\rs{then explain how we do $\Gamma$-calibration--here's the text you wrote before}: We compute the partial identification bounds for vaccination outcomes for multiple values of $\Gamma$ ($\Gamma$ = 1.0, 1.1, ..., 2.4) according to the binary RU regression method, utilizing a linear base predictor for the first quantile regression step. For each value of $\Gamma$, we find the partial identification bounds for each state-level vaccination outcome at two week intervals from March 2021 to August 2021. We then find the $\Gamma$ that minimizes the MSE between the lower partial identification bound and the CDC ground truth at the subgroup level. 

We bootstrap over the test set $B=1000$ times to obtain 95\% confidence intervals on our estimates. In each bootstrapping iteration, we resample the test set with replacement to match its original size. We pass each bootstrapped test set through the model and compute subgroup means to generate our predictions. This is done by taking an average across all respondents in a particular state, weighted according to the post-stratification weights. We then compute the relevant metrics, such as Micro-Geographic Level MAE, and aggregate across all sets to obtain our confidence intervals. 

\amy{Also, is there somewhere I should more explicitly define the three categories of predictions (I'm guessing this will be in results?) and the subsequent MSE, MAE metrics?} \rs{we can define it in the main text!}

 %\amycomm{verify if, for finding argmin Gamma, CTIS is compared to the CDC ground truths at the county (or state) level}

\end{section}


\end{document}

%%% Local Variables:
%%% mode: latex
%%% TeX-master: t
%%% End:
